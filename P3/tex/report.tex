\documentclass[14pt,a4paper]{extarticle}%
\usepackage{amsthm}
\usepackage{amsmath}
\usepackage{amsfonts}
\usepackage{amssymb}
\usepackage{setspace}
\usepackage{listings}
\usepackage{graphicx}
\usepackage{indentfirst}
\usepackage{booktabs}
\usepackage[normalem]{ulem}
\usepackage[T2A]{fontenc}
\usepackage[utf8]{inputenc}
\usepackage[english,russian]{babel}
%-------------------------------------------
\setlength{\textwidth}{7.0in}
\setlength{\oddsidemargin}{-0.35in}
\setlength{\topmargin}{-0.5in}
\setlength{\textheight}{9.0in}
\setlength{\parindent}{0.3in}
\graphicspath{{../plot/}}


\newtheorem{theorem}{Theorem}
\newtheorem{task}[theorem]{Задача}
\addto\captionsrussian{\renewcommand*{\proofname}{Решение}}

\onehalfspacing


\begin{document}

\begin{titlepage}
  \begin{center}
    МИНОБРНАУКИ РОССИИ\\
    САНКТ-ПЕТЕРБУРГСКИЙ ГОСУДАРСТВЕННЫЙ\\
    ЭЛЕКТРОТЕХНИЧЕСКИЙ УНИВЕРСИТЕТ\\
    <<ЛЭТИ>> ИМ. В. И. ЛЕНИНА (УЛЬЯНОВА)\\
    Кафедра МО ЭВМ

    \vspace{4cm}

    ОТЧЕТ\\
    по практической работе №3\\
    по дисциплине <<Теория принятия решений>>\\
    Тема: Игры с природой. Использование вероятностных характеристик в задачах принятия решений.
    \vfill

    \begin{tabular}{ c c c }
      Студент гр. 8303 & \uline{\hspace{3cm}} & Гришин К. И. \\[1cm]
      Преподаватель    & \uline{\hspace{3cm}} & Попова Е. В. \\
    \end{tabular}
    
    \vfill
    Санкт-Петербург\\
    2022
  \end{center}
\end{titlepage}


\section{Цель работы}
Познакомиться с оптимизационными критериями, используемыми при игре
с одним игроком. Применить вероятностные характеристики для решения
задач принятия решений с наличием неопределенности.

\section{Основные теоретические положения}
Существует неопределенность, не связанная с осознанным противодействием
противника, а возникающая в связи с недостаточной информированностью лица,
принимающего решение (ЛПР) об объективных условиях, в которых будет
приниматься решение. В математической модели присутствует «природа»,
и заранее неизвестно её состояние во время принятия решения, игра при
этом называется игрой с природой. Осознанно действует только один игрок.
Природа не противник, принимает какое-то состояние, не преследует цели,
безразлична к результату игры. 

В платежной матрице по строкам реализуются стратегии игрока, а по
столбцам --- множество состояний противоположной стороны. Элементы
столбцов не являются проигрышами природы при соответствующих её
состояниях. Задача выбора игроком чистой или смешанной стратегии
проще так как отсутствует противодействие, но сложнее так как существует
наличие неопределенности, связанное с дефицитом осведомленности игрока о
характере проявления состояний природы.

\section{Задание}
Вариант 24

Необходимо:

Определить оптимальную стратегию сгенерированной платежной матрицы по критерию <<максимакс>>.

Найти вероятностные характеристики задач.


\pagebreak

\section{Выполнение работы}
\subsection{Оптимальная стратегия платежной матрицы}

Генерация случайной $10 \times 10$ матрицы со значениями от $1/4$ до $4$
\begin{lstlisting}[language=Python]
  import numpy as np
  payoff = np.random.uniform(low=1/4, high=4, size=(10, 10))
\end{lstlisting}

\vspace{.5cm}

Поиск максимального выигрыша в каждой стратегии
\begin{lstlisting}[language=Python]
  import numpy as np
  maxes = np.apply_along_axis(np.max, 1, payoff)
\end{lstlisting}

\vspace{.5cm}

Поиск <<максимакса>>
\begin{lstlisting}[language=Python]
  maximax, strategy = maxes.max(), maxes.argmax()
\end{lstlisting}

\vspace{.5cm}
Вывод программы:

\begin{tabular}{lrrrrrrrrrrr}
  \toprule
  {} &  0.119 &  0.032 &  0.129 &  0.122 &  0.115 &  0.121 &  0.092 &  0.095 &  0.033 &  0.143 &    max \\
  \midrule
  0 &  1.466 &  3.464 &  1.625 &  2.903 &  3.342 &  1.309 &  3.356 &  1.075 &  0.620 &  3.232 &  3.464 \\
  1 &  2.920 &  0.608 &  3.057 &  1.522 &  3.719 &  1.345 &  1.808 &  0.458 &  0.668 &  1.588 &  3.719 \\
  2 &  3.806 &  2.901 &  0.903 &  3.554 &  0.308 &  3.020 &  2.272 &  0.466 &  2.694 &  1.108 &  3.806 \\
  3 &  0.561 &  1.188 &  1.838 &  3.920 &  0.389 &  1.892 &  2.057 &  1.341 &  0.596 &  2.277 &  3.920 \\
  4 &  1.098 &  1.515 &  2.295 &  0.339 &  0.906 &  2.797 &  2.625 &  3.208 &  1.829 &  2.953 &  3.208 \\
  5 &  1.085 &  3.870 &  1.581 &  0.674 &  1.738 &  1.922 &  1.288 &  1.109 &  0.447 &  0.788 &  3.870 \\
  6 &  1.640 &  2.121 &  3.930 &  0.601 &  1.604 &  1.854 &  3.823 &  0.804 &  1.109 &  3.437 &  3.930 \\
  7 &  1.754 &  2.700 &  2.472 &  1.940 &  3.358 &  3.898 &  0.919 &  1.042 &  2.553 &  1.207 &  3.898 \\
  8 &  2.965 &  1.527 &  3.944 &  0.885 &  1.218 &  1.015 &  2.893 &  1.072 &  2.440 &  1.455 &  3.944 \\
  9 &  3.817 &  3.518 &  1.612 &  2.762 &  2.891 &  3.003 &  0.433 &  0.709 &  2.557 &  3.988 &  3.988 \\
  \bottomrule
\end{tabular}

maximax=3.988

strategy=9

\section{Вероятностные характеристики задач}
\begin{task}
  В первой урне содержится 2 белых и 3 черных шара,  во второй ---
  3 белых и 2 черных шара. Из первой урны во вторую перекладывают
  один шар, шары перемешивают, а затем извлекают один шар.
  Какова вероятность извлечь при этом черный шар из второй урны
  (решение о составе урн)?
\end{task}

\begin{proof}\hfill\par
  $P(B_1) = 3/5$ --- вероятность достать из первой урны черный шар.

  $P(B_2) = 2/5$ --- вероятность достать из первой урны белый шар.
  
  $P_{B_1}(A) = 1/2$ --- вероятность достать из второй урны черный
  шар при условии, что из первой взят черный

  $P_{B_2}(A) = 1/3$ --- вероятность достать из второй урны черный
  шар при условии, что из первой взят белый

  Необходимо найти полную вероятность того, что из второй урны будет
  вынут черный шар.

  \[
    P(A) = P(B_1)P_{B_1}(A) + P(B_2)P_{B_2}(A)
  \]

  \[
    P(A) = \frac{3}{5}\cdot\frac{1}{2} + \frac{2}{5}\cdot\frac{1}{3}
    = \frac{13}{30} \approx 0.43
  \]
\end{proof}

\begin{task}
  Вероятность того, что при составлении бухгалтерского баланса
  допущена ошибка, равна 0,3. Аудитору представлены 2 баланса.
  Составить закон распределения случайной величины Х- числа
  отрицательных заключений на проверяемые балансы. Найти
  математическое ожидание, дисперсию и среднее квадратическое
  отклонение этой случайной величины.
\end{task}

\begin{proof}
  Возможные исходы из двух балансов:
  \begin{itemize}
    \setlength\itemsep{-.4cm}
    \item Допущена ошибка в обоих балансах, $p = 0.09$
    \item Оба баланса верны, $p = 0.49$
    \item Один из балансов ошибочный, $p = 0.42$
  \end{itemize}

  Таблица распределения числа отрицательных заключений

  \begin{center}
    \begin{tabular}{rccc}
      \toprule
      X & 2    & 1    & 0 \\
      \midrule
      p & 0.09 & 0.42 & 0.49 \\
      \bottomrule
    \end{tabular}
  \end{center}

  \[
    \begin{aligned}
      M &= 2\cdot0.09 + 1\cdot0.42 + 0\cdot0.49 = 0.6 \\
      D &= 4\cdot0.09 + 1\cdot0.42 + 0\cdot0.49 - 0.6^2 = 0.42 \\
      \sigma &= \sqrt{0.42} = 0.648
    \end{aligned}
  \]
\end{proof}


\begin{task}
  Вероятность наступления некоторого события при каждом из 1500
  испытаний равна 0.4. Какова вероятность того, что относительная
  частота отклонится от вероятности по абсолютной величине меньше,
  чем на 0.02.
\end{task}

\begin{proof}
  \[
    P(\left\lvert \frac{m}{n} - p \right\rvert \leq 0.02)
    \cong 2\Phi\left(0.02\sqrt{\frac{1500}{0.24}}\right)
  \]

  \[
    \Phi\left(0.02\sqrt{\frac{1500}{0.24}}\right) \approx \Phi(1.58) \approx 0.4429
  \]

  \[
    P \approx 2 \cdot 0.4429 = 0.8858
  \]
\end{proof}

\section{Вывод}
В ходе выполнения работы были изучены оптимизационные критерии выбора
наилучшей стратегии платежной матрицы, используемые при игре с одним игроком.

Также изучены и найдены вероятностные характеристики различных задач с наличием неопределенности.

\end{document}